% !TEX TS-program = pdflatex
% !TEX encoding = UTF-8 Unicode

% This is a simple template for a LaTeX document using the "article" class.
% See "book", "report", "letter" for other types of document.

\documentclass[20pt]{article} % use larger type; default would be 10pt

\usepackage[utf8]{inputenc} % set input encoding (not needed with XeLaTeX)

%%% Examples of Article customizations
% These packages are optional, depending whether you want the features they provide.
% See the LaTeX Companion or other references for full information.

%%% PAGE DIMENSIONS
\usepackage{geometry} % to change the page dimensions
\geometry{a4paper} % or letterpaper (US) or a5paper or....
% \geometry{margin=2in} % for example, change the margins to 2 inches all round
% \geometry{landscape} % set up the page for landscape
%   read geometry.pdf for detailed page layout information

\usepackage{graphicx} % support the \includegraphics command and options

% \usepackage[parfill]{parskip} % Activate to begin paragraphs with an empty line rather than an indent

%%% PACKAGES
\usepackage{booktabs} % for much better looking tables
\usepackage{array} % for better arrays (eg matrices) in maths
\usepackage{paralist} % very flexible & customisable lists (eg. enumerate/itemize, etc.)
\usepackage{verbatim} % adds environment for commenting out blocks of text & for better verbatim
%\usepackage{subfig} % make it possible to include more than one captioned figure/table in a single float
\usepackage{mathtools}
\usepackage{graphicx} % supports images in latex
% These packages are all incorporated in the memoir class to one degree or another...

\usepackage{graphicx}
\usepackage{subcaption}

%%% Other stuff
\DeclarePairedDelimiter\ceil{\lceil}{\rceil}
\DeclarePairedDelimiter\floor{\lfloor}{\rfloor}

%%% HEADERS & FOOTERS
\usepackage{fancyhdr} % This should be set AFTER setting up the page geometry
\pagestyle{fancy} % options: empty , plain , fancy
\renewcommand{\headrulewidth}{0pt} % customise the layout...
\lhead{}\chead{}\rhead{}
\lfoot{}\cfoot{\thepage}\rfoot{}

%%% SECTION TITLE APPEARANCE
\usepackage{sectsty}
\allsectionsfont{\sffamily\mdseries\upshape} % (See the fntguide.pdf for font help)
% (This matches ConTeXt defaults)

%%% ToC (table of contents) APPEARANCE
\usepackage[nottoc,notlof,notlot]{tocbibind} % Put the bibliography in the ToC
\usepackage[titles,subfigure]{tocloft} % Alter the style of the Table of Contents
\renewcommand{\cftsecfont}{\rmfamily\mdseries\upshape}
\renewcommand{\cftsecpagefont}{\rmfamily\mdseries\upshape} % No bold!

%%% Code syntax highliting
\usepackage{listings}
%\begin{lstlisting}[language=java]
%\end{lstlisting}

%%% graphics path \graphicspath{{./HW5}}

%%% END Article customizations

%%% nice things to keep around

% \noindent\rule{2cm}{0.4pt} 
%%% puts a small horizontal line

% \mathcal{O} 
%%% big O notation

%%% The "real" document content comes below...

\title{Computational Statistics Homework 1}
\author{Liam Dillingham}
%\date{} % Activate to display a given date or no date (if empty),
         % otherwise the current date is printed 

\begin{document}
\maketitle

\section{Question 1} 
The function $p(x) = sin(x)$ is a density for $x \in (0, \pi / 2)$.
\subsection{Describe an inverse CDF transform method to sample random variables with this density.  Plot the histogram and the true density for visual verification.}

\subsection{Set up a rejection sampling method to sample from $p(x)$ using a proposal density $g(x)$. Plot the histogram and the true density for visual verification}
\textit{Hint: You can take $g(x)$ as the uniform density on the interval $(0,\pi / 2)$}. \\

\section{Question 2}
Determine a method to draw samples from the distribution with PDF: $f(x) \propto exp(-x^{4} / 12)$, for $x \in \!R$.  Turn in derivation and code. Plot histogram and true density for visual verification. \\ \noindent\rule{2cm}{0.4pt} \\


\section{Question 3}
Suppose that $X \in \!R^{nxn}$ is a random matrix with independent $\mathcal{N}(0,1)$ entries. Let $\ell_1$ be the smallest Eigenvalue of $S_n = X^{T}X/n$ and let $Y = n\ell_1$.  Edelman showed that as $n \rightarrow \infty$, the PDF of $Y$ approaches \\ 
{\Large $f(y) = \frac{1+\sqrt{y}}{2\sqrt{y}}e^{-(y/2 + \sqrt{y})}$, $0<y<\infty$ } 
\subsection{Develop a method to sample $Y$ from the density $f(y)$ given in the equation above. Show derivation and code. Plot histogram and true density for visual verification.}

\subsection{Test method by estimating $\!E(log(Y))$ by simple Monte Carlo, and giving a $99\%$ confidence interval. Edelman found that the answer was roughly -1.68788.}


\end{document}




