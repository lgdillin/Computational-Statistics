% !TEX TS-program = pdflatex
% !TEX encoding = UTF-8 Unicode

% This is a simple template for a LaTeX document using the "article" class.
% See "book", "report", "letter" for other types of document.

\documentclass[20pt]{article} % use larger type; default would be 10pt

\usepackage[utf8]{inputenc} % set input encoding (not needed with XeLaTeX)

%%% Examples of Article customizations
% These packages are optional, depending whether you want the features they provide.
% See the LaTeX Companion or other references for full information.

%%% PAGE DIMENSIONS
\usepackage{geometry} % to change the page dimensions
\geometry{a4paper} % or letterpaper (US) or a5paper or....
% \geometry{margin=2in} % for example, change the margins to 2 inches all round
% \geometry{landscape} % set up the page for landscape
%   read geometry.pdf for detailed page layout information

\usepackage{graphicx} % support the \includegraphics command and options

% \usepackage[parfill]{parskip} % Activate to begin paragraphs with an empty line rather than an indent

%%% PACKAGES
\usepackage{booktabs} % for much better looking tables
\usepackage{array} % for better arrays (eg matrices) in maths
\usepackage{paralist} % very flexible & customisable lists (eg. enumerate/itemize, etc.)
\usepackage{verbatim} % adds environment for commenting out blocks of text & for better verbatim
%\usepackage{subfig} % make it possible to include more than one captioned figure/table in a single float
\usepackage{mathtools}
\usepackage{graphicx} % supports images in latex
% These packages are all incorporated in the memoir class to one degree or another...

\usepackage{graphicx}
\usepackage{subcaption}

%%% Other stuff
\DeclarePairedDelimiter\ceil{\lceil}{\rceil}
\DeclarePairedDelimiter\floor{\lfloor}{\rfloor}

%%% HEADERS & FOOTERS
\usepackage{fancyhdr} % This should be set AFTER setting up the page geometry
\pagestyle{fancy} % options: empty , plain , fancy
\renewcommand{\headrulewidth}{0pt} % customise the layout...
\lhead{}\chead{}\rhead{}
\lfoot{}\cfoot{\thepage}\rfoot{}

%%% SECTION TITLE APPEARANCE
\usepackage{sectsty}
\allsectionsfont{\sffamily\mdseries\upshape} % (See the fntguide.pdf for font help)
% (This matches ConTeXt defaults)

%%% ToC (table of contents) APPEARANCE
\usepackage[nottoc,notlof,notlot]{tocbibind} % Put the bibliography in the ToC
\usepackage[titles,subfigure]{tocloft} % Alter the style of the Table of Contents
\renewcommand{\cftsecfont}{\rmfamily\mdseries\upshape}
\renewcommand{\cftsecpagefont}{\rmfamily\mdseries\upshape} % No bold!

%%% Code syntax highliting
\usepackage{listings}
%\begin{lstlisting}[language=java]
%\end{lstlisting}

%%% graphics path \graphicspath{{./HW5}}

%%% END Article customizations

%%% nice things to keep around

% \noindent\rule{2cm}{0.4pt} 
%%% puts a small horizontal line

% \mathcal{O} 
%%% big O notation

%%% The "real" document content comes below...

\title{Computational Statistics Homework 1}
\author{Liam Dillingham}
%\date{} % Activate to display a given date or no date (if empty),
         % otherwise the current date is printed 

\begin{document}
\maketitle

\section{Question 1} 
Create an R program which does the following in sequence: \\

\subsection{Create a vector $x$ which contains the numbers $[9,10,11,12,13,14,15,16]$.}
\begin{lstlisting}[language=R]
> x = c(9,10,11,12,13,14,15,16)
\end{lstlisting}

\subsection{Print the last 3 elements of $x$.}
\begin{lstlisting}[language=R]
> x[(length(x)-2):length(x)]
[1] 14 15 16
\end{lstlisting}

\subsection{Print out all the even numbers in $x$.}
\begin{lstlisting}[language=R]
> x[lapply(x, "%%", 2) == 0]
[1] 10 12 14 16
\end{lstlisting}

\subsection{Delete all the even numbers form $x$ and print the resulting list}
\begin{lstlisting}[language=R]
> x = x[!x %in% x[lapply(x, "%%", 2) == 0]]
> x
[1]  9 11 13 15
\end{lstlisting}

\newpage
\section{Question 2}
Write an R program which takes as input an integer $n$ and computes $\sum_{i=1}^{n} (1/2)^{i}$ in three ways: using a $for$ loop, using a $while$ loop, and without a loop (i.e. an analytic function). Print out the results for all three.  What happens when $n$ is very large? do you have any suggerstions on how to make your program more robust to errors when $n$ is large? \\

Here is the code:

\begin{lstlisting}[language=R]
func = function(n, mode) {
  
  result = 0
  if(mode == 1) {
    for(i in 1:n) {
      result = result + 0.5 ^ i
    }
    return(result)
  } else if(mode == 2) {
    i = 1
    while(i <= n) {
      result = result + 0.5 ^ i
      i = i + 1
    }
    return(result)
  } else {
    result = 1 - (0.5 ^ n)
    return(result)
  }
}
\end{lstlisting}

\section{Question 3}
Show true the following: 

\subsection{$x^{3}$ is $\mathcal{O}(x^{3})$ and $\Theta(x^{3})$ but not $\Theta(x^{4})$.}

$x^{3}$ is $\mathcal{O}(x^{3})$ as there exists a constant $c$ (i.e. 1) such that $x^{3} \leq cx^{3}$  $\forall x \geq x_0$, where $x_0$ is some arbitrary but fixed $x$. Also,  $x^{3}$ is $\Theta(x^{3})$ since there exists constants $c_1$ and $c_2$ (i.e. 1 and 1) such that $c_1x^{3} \leq x^{3} \leq c_2x^{3}$  $\forall x \geq x_0$.  However, $x^{3}$ is not $\Theta(x^{4})$ as the distance between $x^{3}$ and $x^{4}$ differs by a factor of $x$ for any given value of $x$.  Therefore, there is no constants $c_1$ and $c_2$ such that $c_1x^{4} \leq x^{3} \leq c_2x^{4}$.

\subsection{For any real constants $a$ and $b>0$, we have $(n+a)^{b} = \Theta(n^{b})$.}

notice that $(n+a)^{b}$ follows the rule of binomial expansion. That is, the largest power of $n$ in the expansion will be $n^{b}$.  as $n\rightarrow +\infty$, The equation become dominated by the largest power of $n$, $n^{b}$.  Thus, $(n+a)^{b} = \Theta(n^{b})$.

\subsection{$(\log(n))^{k} = \mathcal{O}(n)$ for any $k$.}

Note that $log_b(x)$ is the inverse operation for $x^{b}$.  Although we do not know the base for the $log(n)$ function in this problem, if we compose the inverse of two functions (logarithmic and exponential) we will essentially get the input $n$ as the result, giving us $\mathcal{O}(n)$.

\subsection{$n/(n+1) = 1 + \mathcal{O}(1/n)$}
note that $\mathcal{O}(1/n)$ implies the asymtotic behavior of $1/n$.  Since $1/n \rightarrow 0$ as $n\rightarrow \infty$, then the equation becomes $n/(n+1) = 1 + 0$.  Then if we apply a limit as $n \rightarrow \infty$ to both sides, we get $1 =1$ which is true.

\subsection{$\sum_{i=0}^{\ceil{\log_2(n)}} 2^{i}$ is $\Theta(n)$.}

Since $i$th iteration of the summation will be twice as much as the $i-1$ iteration, and will be greater than the total sum of the $0 \rightarrow i-1$ sum.  Since the function is doubling for each increase in $n$, then we can say the function is roughly growing at a rate of $2n$, which is $\Theta(n)$.


\section{Question 4}

\subsection{Implement the bubble sort algorithm taught in the course}
\begin{lstlisting}[language=R]
bubblesort = function(x) {
  n = length(x)
  swapped = TRUE
  
  while(swapped) {
    swapped = FALSE
    for(i in 2:n) {
      if(x[i-1] > x[i]) {
        # swap 
        temp = x[i-1]
        x[i-1] = x[i]
        x[i] = temp
        swapped = TRUE
      }
    }
  }
  return(x)
}
\end{lstlisting}

\subsection{Compare the CPU times for mergesort and bubble sort for a sample size of 1000 normally-distributed numbers}

Bubblesort time: $1.3s$ \\ 
Mergesort time:  $0.07s$

\subsection{Calculate the sample mean and variance for the cpu times after 100 iterations}
Mean Mergesort time: $0.054s$ \\
Mergesort time variance: $0.000230s$ \\
Mean Bubblesort time: $1.252s$ \\
Bubblesort time: $0.0028s$ \\

This appears to be consistent with my predictions. $\mathcal{O}(n\log(n))$ is much more efficient than bubblesort's $\mathcal{O}(n^{2})$ time.  
\end{document}




